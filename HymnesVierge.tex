\documentclass[%%
a5paper%                       Taille de page.
,11pt%                         Taille de police.
,DIV=15%                       Plus grand => des marges plus petites.
,titlepage=on%                 Faut-il une page de titre ?
%,headings=optiontoheadandtoc%  Effet des paramètres optionnels de section.
%,headings=small%
,parskip=false%
,openany%
]{scrbook}
\renewcommand*\partheademptypage{\thispagestyle{empty}}
\newcounter{facteur}\setcounter{facteur}{17}%%%%%%%%%%%%%% Paramètre pour la taille globale des partitions. par défaut~: 17
%\usepackage{geometry}
\usepackage{gredoc,mudoc,lyluatex}
\usepackage{pdfpages,transparent,array,ltablex}
\usepackage{framed}

%%%%%%%%%%%%%%%%%%%%%%% Paramètres variables %%%%%%%%%%%%%%%%%%%%%%%%%%%%%%%%%%%%%%%%%%%%%%%%%%
%%% Taille des partitions grégoriennes.                                                      %%
\grechangedim{overhepisemalowshift}{.7mm}{scalable}
\grechangedim{hepisemamiddleshift}{1.4mm}{scalable}
\grechangedim{overhepisemahighshift}{2.1mm}{scalable}
\grechangedim{vepisemahighshift}{2.1mm}{scalable}
%\grechangestafflinethickness{50} %%% epaisseur des lignes
\grechangestaffsize{\value{facteur}}%%%%% 
%%%%%%%%%%%%%%%%%%%%%%%%%%%%%%%%%%%%%%%%%%%%%%%%%%%%%%%%%%%%%%%%%%%%%%%%%%%%%%%%%%%%%%%%%%%%%%%
% Par souci de clarté, la définition des commandes est reportée dans un document annexe.

\addtolength{\voffset}{2mm}\addtolength{\headsep}{-2mm}
\setlength{\extrarowheight}{2mm}
\addto\captionsfrench{%
  \renewcommand{\indexname}{Index des chants}%
}

\pdfcompresslevel=9

\newcommand{\lieu}[1]{\hfill\linebreak[3]\hspace*{\stretch{1}}\nolinebreak\mbox{\emph{(#1)}}}

\newcommand{\commandement}[1]{\noindent\textbf{#1}}

\newcommand{\schola}[1]{}\newcommand{\foule}[1]{#1}
\providecommand{\dest}{foule}%

\newcommand{\bgimage}[1]{%image d'arrière plan
    \raisebox{-.45\paperheight}[0pt][0pt]{%
      \transparent{0.3}%
      \includegraphics[width=.7\paperwidth,height=.7\paperheight,keepaspectratio=true]{img/#1}%
      }%
}

\def\arraystretch{1.2}

\newcommand{\reponsegras}[2]{
    \versio{\textbf{#1}}{{#2}}
}
\newcommand{\imagecentre}[2]{
\begin{center} \includegraphics[height=#1]{img/#2} \end{center}}

\title{Jubilé de Notre-Dame de Fontpeyrine}
\date{}

\let\oldaddchap\addchap
\def\addchap#1{\oldaddchap{#1}\markright{Mariage}}

\def\blindsection#1{\markright{#1}\addcontentsline{toc}{section}{#1}}
%%%%%%%%%%%%%%%%%%%%%%%%%%%%%%%%%%%%%%%%%%%%%%%%%%%%%%%%%%%%%%%%%%%%%%%%%%%%%%%%
%%%%%%%%%%%%%%%%%%%%%%%%%%%%%%%%%%%%%%%%%%%%%%%%%%%%%%%%%%%%%%%%%%%%%%%%%%%%%%%%
%%%%%%%%%%%%%%%%%%%%%%%%%%%%%%%%%%%%%%%%%%%%%%%%%%%%%%%%%%%%%%%%%%%%%%%%%%%%%%%%

\title{%\centrer
{Grandes Antiennes de l'Avent}}
\author{ou Antiennes O}
\date{du 17 au 23 décembre}

\makeindex
\definecolor{rubrum}{rgb}{.6,0,0}
\def\rubrum{\color{black}}%%%%%%%mettre"\def\rubrum{\color{rubrum}}" pour avoir le texte adéquat en rouge
\def\nigra{\color{black}}
%    \redlines
%    \definecolor{gregoriocolor}{rgb}{.6,0,0}
%
%\let\red\rubrum

\begin{document}

%    \newfontfamily\lettrines[Scale=1.3]{LettrinesPro800}
 %   \def\gretextformat#1{{\fontsize{\taillepolice}{\taillepolice}\selectfont #1}}
  %  \def\greinitialformat#1{{\lettrines #1}}
    

\newcommand{\ligne}[2]{
\begin{center}
\greseparator{#1}{#2}
\end{center}
}
\newcommand{\marubrique}{\rubrica{Magnificat page \pageref{section:magnificat}. Verset et oraisons page \pageref{section:oraisons}}}

\thispagestyle{empty}
\maketitle
\thispagestyle{empty}

\titre{17 décembre}
\vulgo{O Sagesse, sortie de la bouche du Très-Haut, qui enveloppez toutes choses d'un pôle à l'autre et les disposez avec force et douceur, venez nous enseigner le chemin de la prudence.}
\cantus{Antienne}{OSapientia}{}{2.D.}
\marubrique

\ligne{2}{10}

\titre{18 décembre}
\vulgo{O Adonaï, guide du peuple d'Israël, qui êtes apparu à Moïse dans le feu du buisson ardent, et lui avez donné vos commandements sur le mont Sinaï, armez votre bras, et venez nous sauver.}
\cantus{Antienne}{OAdonai}{}{2.D.}
\marubrique
\ligne{2}{10}

\titre{19 décembre}
\vulgo{O Fils de la race de Jessé, signe dressé devant les peuples, vous devant qui les souverains resteront silencieux, vous que les peuples appelleront au secours, délivrez-nous, venez, ne tardez plus !}
\cantus{Antienne}{ORadixIesse}{}{2.D.}
\marubrique
\ligne{2}{10}

\titre{20 décembre}
\vulgo{O Clef de la cité de David, sceptre du royaume d'Israël, vous ouvrez, et personne alors ne peut fermer ; vous fermez, et personne ne peut ouvrir ; venez, faites sortir du cachot le prisonnier établi dans les ténèbres et la nuit de la mort.}
\cantus{Antienne}{OClavisDavid}{}{2.D.}
\marubrique
\ligne{2}{10}

\titre{21 décembre}
\vulgo{O Orient, splendeur de la Lumière éternelle, Soleil de justice, venez, illuminez ceux qui sont assis dans les ténèbres et la nuit de la mort.}
\cantus{Antienne}{OOriens}{}{2.D.}
\marubrique

\ligne{2}{10}

\titre{22 décembre}
\vulgo{O Roi des nations, objet de leur désir, clef de voûte qui unissez les peuples opposés, venez sauver l'homme que vous avez façonné d'argile.}
\cantus{Antienne}{ORexGentium}{}{2.D.}
\marubrique
\ligne{2}{10}

\titre{23 décembre}
\vulgo{O Emmanuel, notre roi et législateur, que tous les peuples attendent comme leur Sauveur, venez nous sauver, Seigneur notre Dieu !}
\cantus{Antienne}{OEmmanuel}{}{2.D.}
\marubrique
\ligne{2}{10}

\newpage
\titre{Magnificat}
\label{section:magnificat}
\cantus{Psaume}{Intonation-Magnificat-2D}{}{2.D.}
\canticum[tonus=per,primus=2,ultimus=0,numerus=\value{numerus}]{Magnificat}
\gloria[tonus=per]
\rubrica{On reprend alors l'antienne, puis le chantre dit le verset :}
\medskip
\label{section:oraisons}

\versio{\vb Roráte cæli désuper et nubes pluant justum.}{\vb Cieux répandez d'en haut votre rosée, et que les nuées fassent pleuvoir le Juste}
\versio{\textbf{\rb Aperiátur terra et gérminet Salvatórem.}}{\textbf{\rb Que la terre s'entrouvre et fasse germer le Sauveur}}
\medskip
\versio{\centering{Orémus}}{\centering{Prions}}

\newfontfamily\malettrine[Scale=0.6]{LettrinesPro800}
\renewcommand{\LettrineFontHook}{\malettrine\color{black}}

\rubrica{avant le 4\ieme dimanche de l'Avent (en dehors des Quatre-Temps)}
\versio{\lettrine{A}{urem} tuam, quǽsumus,
Dómine, précibus nostris
accómmoda: et mentis nostræ
ténebras, grátia tuæ visitatiónis
illústra: Qui vivis.}
{\lettrine{S}{eigneur}, prêtez l’oreille à nos prières : et quand vous nous ferez la grâce de venir parmi nous, apportez votre lumière dans l’obscurité de nos âmes.}

\rubrica{le mercredi des Quatre-Temps :}
\versio{\lettrine{P}{r{\'æ}sta}, quǽsumus, omnípotens
Deus: ut re\-dem\-pti\-ónis
nostræ ventúra solémnitas et
præséntis nobis vitæ subsídia
cónferat, et ætérnæ beatitúdinis
praémia largiátur. Per Dóminum}{\lettrine{F}{aites}, nous vous le demandons, Seigneur : que la solennité approchant de notre rédemption nous apporte avec elle les grâces nécessaires pour la vie présente et nous obtienne la récompense du bonheur éternel.}

\rubrica{le vendredi des Quatre-Temps :}
\versio{\lettrine{E}{xcita}, quǽsumus, Dómine, poténtiam tuam, et veni : ut hi, qui in tua pietáte confídunt, ab omni cítius adversitáte liberéntur : Qui vivis.}
{\lettrine{E}{xcitez} votre puissance, Seigneur, et venez, pour que vos fidèles confiants en votre bonté, soient très vite délivrés de tout ce qui leur fait obstacle.}


\rubrica{à partir du samedi avant le 4\ieme dimanche de l'Avent}
\versio{\lettrine{E}{xcita}, quǽsumus, Dómine, poténtiam tuam, et veni : et magna nobis virtúte succúrre ; ut per auxílium grátiæ tuæ, quod nostra peccáta præpédiunt, indulgéntiæ tuæ propitiatiónis accéleret : Qui vivis et regnas.}
{\lettrine{E}{xcitez}, Seigneur, votre puissance et venez : donnez-nous le secours de votre force infinie, et qu’avec l’aide de votre grâce, votre indulgente bonté nous accorde sans délai ce que retardent nos péchés.}
\end{document}