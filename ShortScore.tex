\documentclass[11pt,a5paper]{article}

%\usepackage{gregoriotex}
\usepackage{fullpage}
\usepackage{libertine}
\usepackage{lettrine}
\usepackage{oldgerm}

%\gresetinitiallines{0}
%\greseteolcustos{manual}%
\pagestyle{empty}

\begin{document}
%{\begin{center} \greseparator{1}{20}\end{center}}
\newfontfamily\malettrine[Scale=0.6]{l800}
\renewcommand{\LettrineFontHook}{\malettrine\color{black}}
\noindent
\lettrine{A}{ce que je vois} le texte est superposé sur deux lignes différentes, ce qui est très gênant pour la lecture, voire même impossible. Tout simplement parce que ce n'es pas prévu pour être lu. Il s'agit d'un bug. Maintenant, il faut savoir d'où vient ce fameux bug. D'où l'intérêt de ces quelques lignes
\noindent%
%\begin{minipage}[t]{4.8cm}%
%	\gabcsnippet{(c4) <sp>V/</sp>. Ky(f)ri(d)e(d) e(d)le(d)i(c)son(d.) (::)}%
%\end{minipage}%
%\begin{minipage}[t]{3.6cm}%
%	\gresetlinecolor{gregoriocolor}%
%	\gresetclef{invisible}%
%	\gabcsnippet{ <sp>R/</sp>.Chris(f)te(d) e(d)le(d)i(c)son(d.)}%
%\end{minipage}%
%\begin{minipage}[t]{4.5cm}%
%	\gresetclef{invisible}%
%	\gabcsnippet{<sp>V/</sp>.(::) Ky(f)ri(d)e(d) e(d)le(d)i(c)son(d.) (::)}%
%\end{minipage}%

\end{document}
